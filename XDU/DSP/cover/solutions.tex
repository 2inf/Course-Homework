\documentclass[cs4size,a4paper]{ctexart}  
\usepackage[left=1in,right=1in,top=1in,bottom=1in]{geometry}
\usepackage{fancyhdr}
\usepackage{listings}
\usepackage{color,xcolor}
\usepackage{tikz}
\usepackage{graphicx}


% 设置\today格式
%\CTEXoptions[today=old]
%\CTEXoptions[today=big]
\CTEXoptions[today=small]

\pagestyle{fancy}

\lhead{Xidian University}
\rhead{Data Structure and Algorithm}
\cfoot{第\thepage 页}


% 定义可能使用到的颜色
\definecolor{CPPLight}  {HTML} {686868}
\definecolor{CPPSteel}  {HTML} {888888}
\definecolor{CPPDark}   {HTML} {262626}
\definecolor{CPPBlue}   {HTML} {4172A3}
\definecolor{CPPGreen}  {HTML} {487818}
\definecolor{CPPBrown}  {HTML} {A07040}
\definecolor{CPPRed}    {HTML} {AD4D3A}
\definecolor{CPPViolet} {HTML} {7040A0}
\definecolor{CPPGray}  {HTML} {B8B8B8}
\lstset{
	columns=fixed,       
	numbers=left,                                        % 在左侧显示行号
	frame=none,                                          % 不显示背景边框
	backgroundcolor=\color[RGB]{245,245,244},            % 设定背景颜色,
	keywordstyle=\color[RGB]{40,40,255},                 % 设定关键字颜色
	numberstyle=\footnotesize\color{darkgray},           % 设定行号格式
	commentstyle=\it\color[RGB]{0,96,96},                % 设置代码注释的格式
	stringstyle=\rmfamily\slshape\color[RGB]{128,0,0},   % 设置字符串格式
	showstringspaces=false,                              % 不显示字符串中的空格
	% language=c++,                                        % 设置语言
	breaklines = true,                                   % 自动换行
	morekeywords={alignas,continute,friend,register,true,alignof,decltype,goto,
		reinterpret_cast,try,asm,defult,if,return,typedef,auto,delete,inline,short,
		typeid,bool,do,int,signed,typename,break,double,long,sizeof,union,case,
		dynamic_cast,mutable,static,unsigned,catch,else,namespace,static_assert,using,
		char,enum,new,static_cast,virtual,char16_t,char32_t,explict,noexcept,struct,
		void,export,nullptr,switch,volatile,class,extern,operator,template,wchar_t,
		const,false,private,this,while,constexpr,float,protected,thread_local,
		const_cast,for,public,throw,std},
	emph={map,set,multimap,multiset,unordered_map,unordered_set,
		unordered_multiset,unordered_multimap,vector,string,list,deque,
		array,stack,forwared_list,iostream,memory,shared_ptr,unique_ptr,
		random,bitset,ostream,istream,cout,cin,endl,move,default_random_engine,
		uniform_int_distribution,iterator,algorithm,functional,bing,numeric,},
	emphstyle=\color{CPPViolet}, 
}

% 字体
\newcommand{\chuhao}{\fontsize{42.2pt}{\baselineskip}\selectfont}
\newcommand{\xiaochu}{\fontsize{36.1pt}{\baselineskip}\selectfont}
\newcommand{\yihao}{\fontsize{26.1pt}{\baselineskip}\selectfont}
\newcommand{\xiaoyi}{\fontsize{24.1pt}{\baselineskip}\selectfont}
\newcommand{\erhao}{\fontsize{22.1pt}{\baselineskip}\selectfont}
\newcommand{\xiaoer}{\fontsize{18.1pt}{\baselineskip}\selectfont}
\newcommand{\sanhao}{\fontsize{16.1pt}{\baselineskip}\selectfont}
\newcommand{\xiaosan}{\fontsize{15.1pt}{\baselineskip}\selectfont}
\newcommand{\sihao}{\fontsize{14.1pt}{\baselineskip}\selectfont}
\newcommand{\xiaosi}{\fontsize{12.1pt}{\baselineskip}\selectfont}
\newcommand{\wuhao}{\fontsize{10.5pt}{\baselineskip}\selectfont}
\newcommand{\xiaowu}{\fontsize{9.0pt}{\baselineskip}\selectfont}
\newcommand{\liuhao}{\fontsize{7.5pt}{\baselineskip}\selectfont}
\newcommand{\xiaoliu}{\fontsize{6.5pt}{\baselineskip}\selectfont}
\newcommand{\qihao}{\fontsize{5.5pt}{\baselineskip}\selectfont}
\newcommand{\bahao}{\fontsize{5.0pt}{\baselineskip}\selectfont}

\newcommand{\HRule}{\rule{\linewidth}{0.5mm}}


\begin{document}
	%\maketitle
	
	\begin{titlepage}
		
		\begin{center}
			
			
			% Upper part of the page
			\includegraphics[width=0.65\textwidth]{figure/logo_xdred}\\[1cm]    
			
			\textsc{\LARGE \bfseries 人工智能学院}\\[1.5cm]
			
			\textsc{\Large 数字信号处理课程实验报告}\\[0.5cm]
			
			% Title
			\HRule \\[0.4cm]
			{ \huge \bfseries 实验 1: 信号的采样}\\[0.4cm]
			
			\HRule \\[1.5cm]
			
			\large \textbf{作者}(排名不分先后): \\[0.5cm]
			% Author and supervisor
			\begin{figure}[htbp]
			\centering
			\begin{minipage}[t]{0.3\textwidth}
				\begin{center} \large
					%\emph{Author:}\\
					\textsc {\kaishu 杨文韬\\18020100245}
			\end{center}
			\end{minipage}
			\hfill
			\begin{minipage}[t]{0.3\textwidth}
				\begin{center} \large
					%\emph{Author:}\\
					\textsc {\kaishu 刘浩\\19069100088}
				\end{center}
			\end{minipage}
			\hfill
			\begin{minipage}[t]{0.3\textwidth}
				\begin{center} \large
					%\emph{Author:}\\
					\textsc {\kaishu 周泽熙\\19069100126}
				\end{center}
			\end{minipage}
			\end{figure}
			
			
			\vfill
			
			% Bottom of the page
			{\large \today}
			
		\end{center}
		
	\end{titlepage}

	\iffalse
	\tableofcontents
	\thispagestyle{empty}
	\clearpage
	\noindent
	%\centering
	\setmainfont{Courier New Bold}
	\setcounter{page}{1}


	\section{题目 2}
	
	\subsection{输入标准}
	
	在一行内表达式以字符串输入,以 \verb|#| 开始和结束,所有表达式中的数\textbf{均为 1 位},运算结果不能超过 $2^{31}$,所有操作包括:加法 \lstinline|+|、减法 \verb|-|、乘法 \verb|*|、除法 \verb|/|、求幂 \verb|^|,输入到 EOF 表示所有输入结束。
	
	\textbf{需要特别注意的地方}:在涉及负数的操作时,例如 $-2$ 不能写成 $(-2)$ 而是写成 $(0-2)$,并且本程序不能求数的负数次幂,在求 $3^{2^{2^2}}$ 形式的多阶幂时,请输入 \verb|#3^(2^(2^2))#| 而不是 \verb|#3^2^2^2#|。
	
	
	\subsection{输出标准}
	
	如果表达式合法,本程序首先会输出中缀表达式转换后得到的后缀表达式,之后输出表达式的结果;若不合法,则会输出错误提示信息。
	
	\subsection{思路}
	
	中缀表达式的求值一般有两种做法,常用的是转化为后缀表达式,另一种不常见的是递归法。本题我用的是第一种思想。
	
	1. 建立一个用于存运算符的栈 st,逐一扫描该中缀表达式中的元素。
	
	(1) 如果遇到一个数,输出该数。
	
	(2) 如果遇到左括号,左括号入栈。
	
	(3) 如果遇到右括号,不断取出栈顶输出,直到栈顶为左括号,然后把左括号出栈。
	
	(4) 如果遇到运算符,只要栈顶符号的优先级不低于新符号,就不断取出栈顶并输出,最后把新符号进栈。优先级:幂运算>乘除>加减>左括号。
	
	2. 依次取出并输出栈中的所有剩余符号,最终输出的序列就是一个与原中缀表达式等价的后缀表达式。
	
	
	幂的计算是利用了 $a^k=\begin{cases}(a^{\frac{k}{2}})^2 \quad k\\a*(a^{\lfloor\frac{k}{2}\rfloor})^2 \quad k \end{cases}$
	
	下面是我验证过的输入数据
	
	\subsection{代码}
	
	\lstinputlisting[language=c++]{code/2.cpp}
	
	\section{题目 6}
	\fi
	
\end{document}