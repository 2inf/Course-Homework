\section{思考与总结}

\subsection{一些思考}

三门问题实际上是一个非常简单的问题,主持人认为概率是 $\frac{1}{2}$ 可能是下面的原因:主持人直观地认为排除一羊后通过观察得换不换赢得汽车的概率为 $\frac{1}{2}$,即两门后一羊一车,随机赢得汽车概率为 $\frac{1}{2}$。有心理学家指出,不转换的行为可以用心理学现象解释为:

\begin{itemize}
	\item \textbf{禀赋效应(Endowment effect)}: 当个人一旦拥有某项物品,那么他对该物品价值的评价要比未拥有之前大大提高。也就是说人们往往会高估已选的中奖概率。 
	\item \textbf{现状偏见(Status quo bias)}: 即使改变现状更有利,也不愿改变的心理。也就是说人们更愿意坚持已经做出的选择。
	\item 在所有其他条件相同的情况下,人们更喜欢通过无为(Stay)而不是行动(Switch)来犯错。
\end{itemize}

还可以扩展到更多门的情况,假设有 $N$ 扇门,其中一扇门后有车,其余全为羊。则在主人打开 $M$ 扇有羊的门后选手选择更换策略赢得汽车的概率为 $\frac{N-1}{N(N-M-1)}$ ,这是因为车在其余 $N-1$ 扇门中的一个的概率为 $\frac{N-1}{N}$ ,车在 $N-1$ 门中的条件下在 $N-M-1$ 个门中选择一个有车概率为 $\frac{1}{N-M-1}$,故若选择更换策略,赢得汽车的概率为 $\frac{N-1}{N}\times \frac{1}{N-M-1}$。对比坚持最初选择赢得汽车的概率 $\frac{1}{N}$ ,采用更换的策略显然赢得汽车的概率更大。然而注意到当 $N$ 很大且 $M$ 很小时,此时即使选择更换赢得汽车的概率仍很小。但是在主持人排除 $M=N-2$ 扇门后,此时更换赢得汽车的概率为 $1-\frac{1}{N}$ ,即当N越大,赢得汽车概率越高。

\subsection{总结}

三门问题给我们的启示是:人们在讨论时可能犯与主持人一样的错误或者忽略掉主持人知道哪一扇门是汽车这一信息而导致结论不一致,很多悖论的出现都是由于条件的限制,例如著名的“贝特朗悖论”\cite{杨培恒1990关于贝特朗奇论的讨论}:“在一个圆内任意选一条弦,这条弦的弦长长于这个圆的内接等边三角形的边长的概率是多少?”,很多学者给出了三个有效但结果不同的论证。另外,笔者在查阅资料时发现,很多人的解答都是事先知道答案之后通过拼凑过程得到与结果一致的结论,而并没有进行严谨的推导。

事实上,三门问题中隐含一个非常重要的原理即\textbf{有效信息的输入能降低不确定性}。第一次选中羊的概率比选中汽车概率大,第二次选择时,如果主持人提供了有效信息,即排除一个一定是羊的门,那么显然应该换门来提高概率,而如果主持人随机选择一扇门打开,那便没有提供有效信息,即换不换门概率不会发生变化。

下面列出了一些可能有用的相关链接

\url{https://en.wikipedia.org/wiki/Monty_Hall_problem}

\url{http://www.montyhallproblem.com/}

\url{http://www.math.ucsd.edu/~crypto/Monty/monty.html}