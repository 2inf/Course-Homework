\section*{\zihao{2} \centering 摘要}

\vskip0.5cm
本文介绍了三门问题的背景,从一位学者的论文中找出错误,即结果与主持人打开有山羊门的概率无关,而是与主持人是否只会选择山羊门打开有关。通过枚举法、贝叶斯公式和蒙特卡罗方法模拟得到了正确的结论,蒙特卡罗方法是一种统计模拟方法,采样越多,越近似最优解,笔者用 MATLAB 自己编写算法实现了蒙特卡罗方法,并对其进行了可视化展示,并得出一致结论:在主持人每次选择羊的门打开的情况下,选手坚持策略赢得车的概率为 $\frac{1}{3}$,而换门策略赢得汽车的概率为 $\frac{2}{3}$,故选手需要改变策略,在文章的最后,给出了一些思考与总结,指出很多悖论的产生都是因为条件限制而引起的。

\textbf{关键词:}  枚举法,条件概率,贝叶斯公式,蒙特卡罗方法,悖论 
% \addcontentsline{toc}{section}{摘要}

% \clearpage
% \section*{\zihao{2} \centering \textbf{Abstract} }
%   %用了Times New Roman字体来美化观感

% Support vector machine (SVM) is a new pattern recognition method developed on the basis of statistical learning theory. It shows many unique advantages in solving small sample, nonlinear and high-dimensional pattern recognition problems. Handwritten digit recognition is one of the research hotspots with high practical value in image processing and pattern recognition. The MNIST database (Mixed National Institute of Standards and Technology database) is a large database of handwritten digits, usually used to train various image processing systems. The database is also widely used for training and testing in the field of machine learning. In this paper, the SVM kernel method is used, and the kernel functions such as RBF kernel, linear kernel, Sigmoid kernel and customized second-order norm kernel are used to realize the handwritten digit recognition function. The results show that the accuracy of model recognition obtained by RBF kernel is more than 98\%, which is more satisfactory than other kernel methods. \\

% \textbf{Key Words:} SVM, Handwritten Digit Recognition, Kernel Function
\addcontentsline{toc}{section}{摘要}




